%% Generated by Sphinx.
\def\sphinxdocclass{report}
\documentclass[letterpaper,10pt,english]{sphinxmanual}
\ifdefined\pdfpxdimen
   \let\sphinxpxdimen\pdfpxdimen\else\newdimen\sphinxpxdimen
\fi \sphinxpxdimen=.75bp\relax

\PassOptionsToPackage{warn}{textcomp}
\usepackage[utf8]{inputenc}
\ifdefined\DeclareUnicodeCharacter
% support both utf8 and utf8x syntaxes
  \ifdefined\DeclareUnicodeCharacterAsOptional
    \def\sphinxDUC#1{\DeclareUnicodeCharacter{"#1}}
  \else
    \let\sphinxDUC\DeclareUnicodeCharacter
  \fi
  \sphinxDUC{00A0}{\nobreakspace}
  \sphinxDUC{2500}{\sphinxunichar{2500}}
  \sphinxDUC{2502}{\sphinxunichar{2502}}
  \sphinxDUC{2514}{\sphinxunichar{2514}}
  \sphinxDUC{251C}{\sphinxunichar{251C}}
  \sphinxDUC{2572}{\textbackslash}
\fi
\usepackage{cmap}
\usepackage[T1]{fontenc}
\usepackage{amsmath,amssymb,amstext}
\usepackage{babel}



\usepackage{times}
\expandafter\ifx\csname T@LGR\endcsname\relax
\else
% LGR was declared as font encoding
  \substitutefont{LGR}{\rmdefault}{cmr}
  \substitutefont{LGR}{\sfdefault}{cmss}
  \substitutefont{LGR}{\ttdefault}{cmtt}
\fi
\expandafter\ifx\csname T@X2\endcsname\relax
  \expandafter\ifx\csname T@T2A\endcsname\relax
  \else
  % T2A was declared as font encoding
    \substitutefont{T2A}{\rmdefault}{cmr}
    \substitutefont{T2A}{\sfdefault}{cmss}
    \substitutefont{T2A}{\ttdefault}{cmtt}
  \fi
\else
% X2 was declared as font encoding
  \substitutefont{X2}{\rmdefault}{cmr}
  \substitutefont{X2}{\sfdefault}{cmss}
  \substitutefont{X2}{\ttdefault}{cmtt}
\fi


\usepackage[Bjarne]{fncychap}
\usepackage{sphinx}

\fvset{fontsize=\small}
\usepackage{geometry}


% Include hyperref last.
\usepackage{hyperref}
% Fix anchor placement for figures with captions.
\usepackage{hypcap}% it must be loaded after hyperref.
% Set up styles of URL: it should be placed after hyperref.
\urlstyle{same}
\addto\captionsenglish{\renewcommand{\contentsname}{Contents:}}

\usepackage{sphinxmessages}
\setcounter{tocdepth}{1}



\title{RADIATE SDK}
\date{Jun 03, 2020}
\release{}
\author{Marcel Sheeny}
\newcommand{\sphinxlogo}{\vbox{}}
\renewcommand{\releasename}{}
\makeindex
\begin{document}

\pagestyle{empty}
\sphinxmaketitle
\pagestyle{plain}
\sphinxtableofcontents
\pagestyle{normal}
\phantomsection\label{\detokenize{index::doc}}



\chapter{radiate\_sdk}
\label{\detokenize{modules:radiate-sdk}}\label{\detokenize{modules::doc}}

\section{radiate module}
\label{\detokenize{radiate:module-radiate}}\label{\detokenize{radiate:radiate-module}}\label{\detokenize{radiate::doc}}\index{radiate (module)@\spxentry{radiate}\spxextra{module}}\index{Sequence (class in radiate)@\spxentry{Sequence}\spxextra{class in radiate}}

\begin{fulllineitems}
\phantomsection\label{\detokenize{radiate:radiate.Sequence}}\pysiglinewithargsret{\sphinxbfcode{\sphinxupquote{class }}\sphinxcode{\sphinxupquote{radiate.}}\sphinxbfcode{\sphinxupquote{Sequence}}}{\emph{sequence\_path}, \emph{config\_file=\textquotesingle{}config/config.yaml\textquotesingle{}}}{}
Bases: \sphinxcode{\sphinxupquote{object}}

This class loads the sequence of RADIATE dataset

\begin{DUlineblock}{0em}
\item[] Example:
\item[] \textgreater{}\textgreater{}\textgreater{} import radiate
\item[] \textgreater{}\textgreater{}\textgreater{} root\_path = ‘path/to/radiate/city\_3\_7/’
\item[] \textgreater{}\textgreater{}\textgreater{} seq = radiate.Sequence(root\_path)
\item[] \textgreater{}\textgreater{}\textgreater{} output = seq.get\_from\_timestamp(seq.init\_timestamp)
\item[] \textgreater{}\textgreater{}\textgreater{} seq.vis\_all(output)
\end{DUlineblock}
\index{cfar() (radiate.Sequence method)@\spxentry{cfar()}\spxextra{radiate.Sequence method}}

\begin{fulllineitems}
\phantomsection\label{\detokenize{radiate:radiate.Sequence.cfar}}\pysiglinewithargsret{\sphinxbfcode{\sphinxupquote{cfar}}}{\emph{x}, \emph{num\_train}, \emph{num\_guard}, \emph{rate\_fa}}{}
Detect peaks with CFAR algorithm.
\begin{quote}\begin{description}
\item[{Parameters}] \leavevmode\begin{itemize}
\item {} 
\sphinxstyleliteralstrong{\sphinxupquote{x}} (\sphinxstyleliteralemphasis{\sphinxupquote{np.array}}) \textendash{} input 1d array

\item {} 
\sphinxstyleliteralstrong{\sphinxupquote{num\_train}} (\sphinxstyleliteralemphasis{\sphinxupquote{int}}) \textendash{} Number of training cells.

\item {} 
\sphinxstyleliteralstrong{\sphinxupquote{num\_guard}} (\sphinxstyleliteralemphasis{\sphinxupquote{int}}) \textendash{} Number of guard cells.

\item {} 
\sphinxstyleliteralstrong{\sphinxupquote{rate\_fa}} (\sphinxstyleliteralemphasis{\sphinxupquote{float}}) \textendash{} False alarm rate.

\end{itemize}

\item[{Returns}] \leavevmode
detected points

\item[{Return type}] \leavevmode
np.array

\end{description}\end{quote}

\end{fulllineitems}

\index{cfar2d() (radiate.Sequence method)@\spxentry{cfar2d()}\spxextra{radiate.Sequence method}}

\begin{fulllineitems}
\phantomsection\label{\detokenize{radiate:radiate.Sequence.cfar2d}}\pysiglinewithargsret{\sphinxbfcode{\sphinxupquote{cfar2d}}}{\emph{x}, \emph{num\_train}, \emph{num\_guard}, \emph{rate\_fa}}{}
Detect peaks with 2D CFAR algorithm in each row.
\begin{quote}\begin{description}
\item[{Parameters}] \leavevmode\begin{itemize}
\item {} 
\sphinxstyleliteralstrong{\sphinxupquote{x}} (\sphinxstyleliteralemphasis{\sphinxupquote{np.array}}) \textendash{} input 2d array

\item {} 
\sphinxstyleliteralstrong{\sphinxupquote{num\_train}} (\sphinxstyleliteralemphasis{\sphinxupquote{int}}) \textendash{} Number of training cells.

\item {} 
\sphinxstyleliteralstrong{\sphinxupquote{num\_guard}} (\sphinxstyleliteralemphasis{\sphinxupquote{int}}) \textendash{} Number of guard cells.

\item {} 
\sphinxstyleliteralstrong{\sphinxupquote{rate\_fa}} (\sphinxstyleliteralemphasis{\sphinxupquote{float}}) \textendash{} False alarm rate.

\end{itemize}

\item[{Returns}] \leavevmode
detected points

\item[{Return type}] \leavevmode
np.array

\end{description}\end{quote}

\end{fulllineitems}

\index{draw\_boundingbox\_rot() (radiate.Sequence method)@\spxentry{draw\_boundingbox\_rot()}\spxextra{radiate.Sequence method}}

\begin{fulllineitems}
\phantomsection\label{\detokenize{radiate:radiate.Sequence.draw_boundingbox_rot}}\pysiglinewithargsret{\sphinxbfcode{\sphinxupquote{draw\_boundingbox\_rot}}}{\emph{im}, \emph{bbox}, \emph{angle}, \emph{color}}{}
\end{fulllineitems}

\index{get\_annotation\_from\_id() (radiate.Sequence method)@\spxentry{get\_annotation\_from\_id()}\spxextra{radiate.Sequence method}}

\begin{fulllineitems}
\phantomsection\label{\detokenize{radiate:radiate.Sequence.get_annotation_from_id}}\pysiglinewithargsret{\sphinxbfcode{\sphinxupquote{get\_annotation\_from\_id}}}{\emph{annotation\_id}}{}
get the annotation from an id
\begin{quote}\begin{description}
\item[{Parameters}] \leavevmode
\sphinxstyleliteralstrong{\sphinxupquote{annotation\_id}} (\sphinxstyleliteralemphasis{\sphinxupquote{int}}) \textendash{} frame id

\item[{Returns}] \leavevmode
list of annotations for the id given as parameter

\item[{Return type}] \leavevmode
list

\end{description}\end{quote}

\end{fulllineitems}

\index{get\_from\_timestamp() (radiate.Sequence method)@\spxentry{get\_from\_timestamp()}\spxextra{radiate.Sequence method}}

\begin{fulllineitems}
\phantomsection\label{\detokenize{radiate:radiate.Sequence.get_from_timestamp}}\pysiglinewithargsret{\sphinxbfcode{\sphinxupquote{get\_from\_timestamp}}}{\emph{t}, \emph{get\_sensors=True}, \emph{get\_annotations=True}}{}
method to get sensor and annotation information from some timestamp
\begin{quote}\begin{description}
\item[{Parameters}] \leavevmode\begin{itemize}
\item {} 
\sphinxstyleliteralstrong{\sphinxupquote{t}} (\sphinxstyleliteralemphasis{\sphinxupquote{float}}) \textendash{} This is the timestamp which access the sensors/annotations

\item {} 
\sphinxstyleliteralstrong{\sphinxupquote{get\_sensors}} (\sphinxstyleliteralemphasis{\sphinxupquote{bool}}\sphinxstyleliteralemphasis{\sphinxupquote{, }}\sphinxstyleliteralemphasis{\sphinxupquote{optional}}) \textendash{} whether to retrieve sensor information, defaults to True

\item {} 
\sphinxstyleliteralstrong{\sphinxupquote{get\_annotations}} (\sphinxstyleliteralemphasis{\sphinxupquote{bool}}\sphinxstyleliteralemphasis{\sphinxupquote{, }}\sphinxstyleliteralemphasis{\sphinxupquote{optional}}) \textendash{} whether to retrieve annotation info, defaults to True

\end{itemize}

\item[{Returns}] \leavevmode
returns a single variable as a dictionary with ‘sensors’ and ‘annotations’ as key

\item[{Return type}] \leavevmode
dict

\end{description}\end{quote}

\end{fulllineitems}

\index{get\_id() (radiate.Sequence method)@\spxentry{get\_id()}\spxextra{radiate.Sequence method}}

\begin{fulllineitems}
\phantomsection\label{\detokenize{radiate:radiate.Sequence.get_id}}\pysiglinewithargsret{\sphinxbfcode{\sphinxupquote{get\_id}}}{\emph{t}, \emph{all\_timestamps}, \emph{time\_offset=0.0}}{}
get the closest id given the timestamp
\begin{quote}\begin{description}
\item[{Parameters}] \leavevmode\begin{itemize}
\item {} 
\sphinxstyleliteralstrong{\sphinxupquote{t}} (\sphinxstyleliteralemphasis{\sphinxupquote{float}}) \textendash{} timestamp in seconds

\item {} 
\sphinxstyleliteralstrong{\sphinxupquote{all\_timestamps}} (\sphinxstyleliteralemphasis{\sphinxupquote{np.array}}) \textendash{} a list with all timestamps

\item {} 
\sphinxstyleliteralstrong{\sphinxupquote{time\_offset}} (\sphinxstyleliteralemphasis{\sphinxupquote{float}}\sphinxstyleliteralemphasis{\sphinxupquote{, }}\sphinxstyleliteralemphasis{\sphinxupquote{optional}}) \textendash{} offset in case there is some unsynchronoised sensor, defaults to 0.0

\end{itemize}

\item[{Returns}] \leavevmode
the closest id

\item[{Return type}] \leavevmode
int

\end{description}\end{quote}

\end{fulllineitems}

\index{get\_lidar\_annotations() (radiate.Sequence method)@\spxentry{get\_lidar\_annotations()}\spxextra{radiate.Sequence method}}

\begin{fulllineitems}
\phantomsection\label{\detokenize{radiate:radiate.Sequence.get_lidar_annotations}}\pysiglinewithargsret{\sphinxbfcode{\sphinxupquote{get\_lidar\_annotations}}}{\emph{id\_radar}, \emph{interp=False}, \emph{t\_c=None}, \emph{t\_r1=None}, \emph{t\_r2=None}}{}
get the annotations in lidar image coordinate frame
\begin{quote}\begin{description}
\item[{Parameters}] \leavevmode\begin{itemize}
\item {} 
\sphinxstyleliteralstrong{\sphinxupquote{id\_radar}} (\sphinxstyleliteralemphasis{\sphinxupquote{int}}) \textendash{} the annotation radar id

\item {} 
\sphinxstyleliteralstrong{\sphinxupquote{interp}} (\sphinxstyleliteralemphasis{\sphinxupquote{bool}}) \textendash{} whether to use interpolation or not

\item {} 
\sphinxstyleliteralstrong{\sphinxupquote{t}} (\sphinxstyleliteralemphasis{\sphinxupquote{float}}) \textendash{} timestamp

\end{itemize}

\item[{Returns}] \leavevmode
the annotations in lidar image coordinate frame

\item[{Return type}] \leavevmode
dict

\end{description}\end{quote}

\end{fulllineitems}

\index{get\_rectfied() (radiate.Sequence method)@\spxentry{get\_rectfied()}\spxextra{radiate.Sequence method}}

\begin{fulllineitems}
\phantomsection\label{\detokenize{radiate:radiate.Sequence.get_rectfied}}\pysiglinewithargsret{\sphinxbfcode{\sphinxupquote{get\_rectfied}}}{\emph{left\_im}, \emph{right\_im}}{}
get the left and right image rectfied
\begin{quote}\begin{description}
\item[{Parameters}] \leavevmode\begin{itemize}
\item {} 
\sphinxstyleliteralstrong{\sphinxupquote{left\_im}} (\sphinxstyleliteralemphasis{\sphinxupquote{np.array}}) \textendash{} raw left image

\item {} 
\sphinxstyleliteralstrong{\sphinxupquote{right\_im}} (\sphinxstyleliteralemphasis{\sphinxupquote{np.array}}) \textendash{} raw right image

\end{itemize}

\item[{Returns}] \leavevmode
tuple (left\_rect, right\_rect, disp\_to\_depth)
WHERE
np.array left\_rect is the rectfied left image
np.array right\_rect is the rectfied right image
np.array disp\_to\_depth is a matrix that converts the disparity values to distance in meters

\item[{Return type}] \leavevmode
tuple

\end{description}\end{quote}

\end{fulllineitems}

\index{lidar\_to\_image() (radiate.Sequence method)@\spxentry{lidar\_to\_image()}\spxextra{radiate.Sequence method}}

\begin{fulllineitems}
\phantomsection\label{\detokenize{radiate:radiate.Sequence.lidar_to_image}}\pysiglinewithargsret{\sphinxbfcode{\sphinxupquote{lidar\_to\_image}}}{\emph{lidar}}{}
Convert an lidar point cloud to an 2d bird’s eye view image
\begin{quote}\begin{description}
\item[{Parameters}] \leavevmode
\sphinxstyleliteralstrong{\sphinxupquote{lidar}} (\sphinxstyleliteralemphasis{\sphinxupquote{np.array}}) \textendash{} lidar point cloud Nx5 (x,y,z, intensity, ring)

\item[{Returns}] \leavevmode
2d bird’s eye image with the lidar information

\item[{Return type}] \leavevmode
np.array

\end{description}\end{quote}

\end{fulllineitems}

\index{load\_timestamp() (radiate.Sequence method)@\spxentry{load\_timestamp()}\spxextra{radiate.Sequence method}}

\begin{fulllineitems}
\phantomsection\label{\detokenize{radiate:radiate.Sequence.load_timestamp}}\pysiglinewithargsret{\sphinxbfcode{\sphinxupquote{load\_timestamp}}}{\emph{timestamp\_path}}{}
load all timestamps from a sensor
\begin{quote}\begin{description}
\item[{Parameters}] \leavevmode
\sphinxstyleliteralstrong{\sphinxupquote{timestamp\_path}} (\sphinxstyleliteralemphasis{\sphinxupquote{string}}) \textendash{} path to text file with all timestamps

\item[{Returns}] \leavevmode
list of all timestamps

\item[{Return type}] \leavevmode
dict

\end{description}\end{quote}

\end{fulllineitems}

\index{overlay\_camera\_lidar() (radiate.Sequence method)@\spxentry{overlay\_camera\_lidar()}\spxextra{radiate.Sequence method}}

\begin{fulllineitems}
\phantomsection\label{\detokenize{radiate:radiate.Sequence.overlay_camera_lidar}}\pysiglinewithargsret{\sphinxbfcode{\sphinxupquote{overlay\_camera\_lidar}}}{\emph{camera}, \emph{lidar}}{}
Method that joins camera and projected lidar in one image for visualisation
\begin{quote}\begin{description}
\item[{Parameters}] \leavevmode\begin{itemize}
\item {} 
\sphinxstyleliteralstrong{\sphinxupquote{camera}} (\sphinxstyleliteralemphasis{\sphinxupquote{np.array}}) \textendash{} camera image

\item {} 
\sphinxstyleliteralstrong{\sphinxupquote{lidar}} (\sphinxstyleliteralemphasis{\sphinxupquote{np.array}}) \textendash{} lidar image with the same size as camera

\end{itemize}

\item[{Returns}] \leavevmode
overlayed image

\item[{Return type}] \leavevmode
np.array

\end{description}\end{quote}

\end{fulllineitems}

\index{project\_bboxes\_to\_camera() (radiate.Sequence method)@\spxentry{project\_bboxes\_to\_camera()}\spxextra{radiate.Sequence method}}

\begin{fulllineitems}
\phantomsection\label{\detokenize{radiate:radiate.Sequence.project_bboxes_to_camera}}\pysiglinewithargsret{\sphinxbfcode{\sphinxupquote{project\_bboxes\_to\_camera}}}{\emph{annotations}, \emph{intrinsict}, \emph{extrinsic}}{}
method to project the bounding boxes to the camera
\begin{quote}\begin{description}
\item[{Parameters}] \leavevmode\begin{itemize}
\item {} 
\sphinxstyleliteralstrong{\sphinxupquote{annotations}} (\sphinxstyleliteralemphasis{\sphinxupquote{list}}) \textendash{} the annotations for the current frame

\item {} 
\sphinxstyleliteralstrong{\sphinxupquote{intrinsict}} (\sphinxstyleliteralemphasis{\sphinxupquote{np.array}}) \textendash{} intrisic camera parameters

\item {} 
\sphinxstyleliteralstrong{\sphinxupquote{extrinsic}} (\sphinxstyleliteralemphasis{\sphinxupquote{np.array}}) \textendash{} extrinsic parameters

\end{itemize}

\item[{Returns}] \leavevmode
dictionary with the list of bbounding boxes with camera coordinate frames

\item[{Return type}] \leavevmode
dict

\end{description}\end{quote}

\end{fulllineitems}

\index{project\_lidar() (radiate.Sequence method)@\spxentry{project\_lidar()}\spxextra{radiate.Sequence method}}

\begin{fulllineitems}
\phantomsection\label{\detokenize{radiate:radiate.Sequence.project_lidar}}\pysiglinewithargsret{\sphinxbfcode{\sphinxupquote{project\_lidar}}}{\emph{lidar}, \emph{lidar\_extrinsics}, \emph{cam\_intrinsic}, \emph{color\_mode=\textquotesingle{}same\textquotesingle{}}}{}
Method to project the lidar into the camera
\begin{quote}\begin{description}
\item[{Parameters}] \leavevmode\begin{itemize}
\item {} 
\sphinxstyleliteralstrong{\sphinxupquote{lidar}} (\sphinxstyleliteralemphasis{\sphinxupquote{np.array}}) \textendash{} lidar point cloud with shape Nx5 (x,y,z,intensity,ring)

\item {} 
\sphinxstyleliteralstrong{\sphinxupquote{lidar\_extrinsics}} (\sphinxstyleliteralemphasis{\sphinxupquote{np.array}}) \textendash{} 4x4 matrix with lidar extrinsic parameters (Rotation
and translations)

\item {} 
\sphinxstyleliteralstrong{\sphinxupquote{cam\_intrinsic}} (\sphinxstyleliteralemphasis{\sphinxupquote{np.array}}) \textendash{} 3x3 matrix with camera intrinsic parameters in the form
{[}{[}fx 0 cx{]},
{[}0 fx cy{]},
{[}0 0 1{]}{]}

\item {} 
\sphinxstyleliteralstrong{\sphinxupquote{color\_mode}} (\sphinxstyleliteralemphasis{\sphinxupquote{string}}) \textendash{} what type of information is going to be representend in the lidar image

\end{itemize}

\end{description}\end{quote}

options: ‘same’ always constant color. ‘pseudo\_distance’: uses a color map to create a psedo
color which refers to the distance. ‘distance’ creates an image with the actual distance as float
\begin{quote}\begin{description}
\item[{Return type}] \leavevmode
np.array

\item[{Returns}] \leavevmode
returns the projected lidar into the respective camera with the same size as the camera

\end{description}\end{quote}

\end{fulllineitems}

\index{read\_lidar() (radiate.Sequence method)@\spxentry{read\_lidar()}\spxextra{radiate.Sequence method}}

\begin{fulllineitems}
\phantomsection\label{\detokenize{radiate:radiate.Sequence.read_lidar}}\pysiglinewithargsret{\sphinxbfcode{\sphinxupquote{read\_lidar}}}{\emph{lidar\_path}}{}
given a lidar raw path returns it lidar point cloud
\begin{quote}\begin{description}
\item[{Parameters}] \leavevmode
\sphinxstyleliteralstrong{\sphinxupquote{lidar\_path}} (\sphinxstyleliteralemphasis{\sphinxupquote{string}}) \textendash{} path to lidar raw point

\item[{Returns}] \leavevmode
lidar point cloud Nx5 (x,y,z,intensity,ring)

\item[{Return type}] \leavevmode
np.array

\end{description}\end{quote}

\end{fulllineitems}

\index{transform\_annotations() (radiate.Sequence method)@\spxentry{transform\_annotations()}\spxextra{radiate.Sequence method}}

\begin{fulllineitems}
\phantomsection\label{\detokenize{radiate:radiate.Sequence.transform_annotations}}\pysiglinewithargsret{\sphinxbfcode{\sphinxupquote{transform\_annotations}}}{\emph{annotations}, \emph{M}}{}
method to transform the annotations to annother coordinate
\begin{quote}\begin{description}
\item[{Parameters}] \leavevmode\begin{itemize}
\item {} 
\sphinxstyleliteralstrong{\sphinxupquote{annotations}} (\sphinxstyleliteralemphasis{\sphinxupquote{list}}) \textendash{} the list of annotations

\item {} 
\sphinxstyleliteralstrong{\sphinxupquote{M}} (\sphinxstyleliteralemphasis{\sphinxupquote{np.array}}) \textendash{} transformation matrix

\end{itemize}

\item[{Returns}] \leavevmode
the list of annotations in another coodinate frame

\item[{Return type}] \leavevmode
list

\end{description}\end{quote}

\end{fulllineitems}

\index{transform\_point\_cloud() (radiate.Sequence method)@\spxentry{transform\_point\_cloud()}\spxextra{radiate.Sequence method}}

\begin{fulllineitems}
\phantomsection\label{\detokenize{radiate:radiate.Sequence.transform_point_cloud}}\pysiglinewithargsret{\sphinxbfcode{\sphinxupquote{transform\_point\_cloud}}}{\emph{pc}, \emph{M}}{}
transform a 3d point cloud to another coordinate frame
\begin{quote}\begin{description}
\item[{Parameters}] \leavevmode\begin{itemize}
\item {} 
\sphinxstyleliteralstrong{\sphinxupquote{pc}} (\sphinxstyleliteralemphasis{\sphinxupquote{np.array}}) \textendash{} point cloud in the form Nx\% (x,y,z,intensity, ring)

\item {} 
\sphinxstyleliteralstrong{\sphinxupquote{M}} (\sphinxstyleliteralemphasis{\sphinxupquote{np.array}}) \textendash{} transformation matrix

\end{itemize}

\item[{Returns}] \leavevmode
transformed point cloud

\item[{Return type}] \leavevmode
np.array

\end{description}\end{quote}

\end{fulllineitems}

\index{vis() (radiate.Sequence method)@\spxentry{vis()}\spxextra{radiate.Sequence method}}

\begin{fulllineitems}
\phantomsection\label{\detokenize{radiate:radiate.Sequence.vis}}\pysiglinewithargsret{\sphinxbfcode{\sphinxupquote{vis}}}{\emph{sensor}, \emph{objects}, \emph{color=None}, \emph{mode=\textquotesingle{}rot\textquotesingle{}}}{}
visualise the sensor and its annotation
\begin{quote}\begin{description}
\item[{Parameters}] \leavevmode\begin{itemize}
\item {} 
\sphinxstyleliteralstrong{\sphinxupquote{sensor}} (\sphinxstyleliteralemphasis{\sphinxupquote{the given sensor}}) \textendash{} 

\item {} 
\sphinxstyleliteralstrong{\sphinxupquote{objects}} (\sphinxstyleliteralemphasis{\sphinxupquote{list of objects}}) \textendash{} np.array

\end{itemize}

\item[{Returns}] \leavevmode
image with the objects overlayed

\item[{Return type}] \leavevmode
np.array

\end{description}\end{quote}

\end{fulllineitems}

\index{vis\_3d\_bbox\_cam() (radiate.Sequence method)@\spxentry{vis\_3d\_bbox\_cam()}\spxextra{radiate.Sequence method}}

\begin{fulllineitems}
\phantomsection\label{\detokenize{radiate:radiate.Sequence.vis_3d_bbox_cam}}\pysiglinewithargsret{\sphinxbfcode{\sphinxupquote{vis\_3d\_bbox\_cam}}}{\emph{image}, \emph{bboxes\_3d}, \emph{pc\_size=0.7}}{}
diplay pseudo 3d bounding box from camera
\begin{quote}\begin{description}
\item[{Parameters}] \leavevmode\begin{itemize}
\item {} 
\sphinxstyleliteralstrong{\sphinxupquote{image}} (\sphinxstyleliteralemphasis{\sphinxupquote{np.array}}) \textendash{} camera which the bounding box is going to be projected

\item {} 
\sphinxstyleliteralstrong{\sphinxupquote{bboxes\_3d}} (\sphinxstyleliteralemphasis{\sphinxupquote{dict}}) \textendash{} list of bounding box information with pseudo\sphinxhyphen{}3d image coordinate frame

\item {} 
\sphinxstyleliteralstrong{\sphinxupquote{pc\_size}} (\sphinxstyleliteralemphasis{\sphinxupquote{float}}) \textendash{} percentage of the size of the bounding box {[}0.0 1.0{]}

\end{itemize}

\item[{Returns}] \leavevmode
camera image with the correspondent bounding boxes

\item[{Return type}] \leavevmode
np.array

\end{description}\end{quote}

\end{fulllineitems}

\index{vis\_all() (radiate.Sequence method)@\spxentry{vis\_all()}\spxextra{radiate.Sequence method}}

\begin{fulllineitems}
\phantomsection\label{\detokenize{radiate:radiate.Sequence.vis_all}}\pysiglinewithargsret{\sphinxbfcode{\sphinxupquote{vis\_all}}}{\emph{output}, \emph{wait\_time=1}}{}
method to diplay all the sensors/annotations
\begin{quote}\begin{description}
\item[{Parameters}] \leavevmode\begin{itemize}
\item {} 
\sphinxstyleliteralstrong{\sphinxupquote{output}} (\sphinxstyleliteralemphasis{\sphinxupquote{dict}}) \textendash{} gets the output from self.get\_from\_timestamp(t)

\item {} 
\sphinxstyleliteralstrong{\sphinxupquote{wait\_time}} (\sphinxstyleliteralemphasis{\sphinxupquote{int}}\sphinxstyleliteralemphasis{\sphinxupquote{, }}\sphinxstyleliteralemphasis{\sphinxupquote{optional}}) \textendash{} how to long to wait until display next frame. 0 means it will wait for any key, defaults to 1

\end{itemize}

\end{description}\end{quote}

\end{fulllineitems}

\index{vis\_bbox\_cam() (radiate.Sequence method)@\spxentry{vis\_bbox\_cam()}\spxextra{radiate.Sequence method}}

\begin{fulllineitems}
\phantomsection\label{\detokenize{radiate:radiate.Sequence.vis_bbox_cam}}\pysiglinewithargsret{\sphinxbfcode{\sphinxupquote{vis\_bbox\_cam}}}{\emph{image}, \emph{bboxes\_3d}, \emph{pc\_size=0.7}}{}
diplay pseudo 2d bounding box from camera
\begin{quote}\begin{description}
\item[{Parameters}] \leavevmode\begin{itemize}
\item {} 
\sphinxstyleliteralstrong{\sphinxupquote{image}} (\sphinxstyleliteralemphasis{\sphinxupquote{np.array}}) \textendash{} camera which the bounding box is going to be projected

\item {} 
\sphinxstyleliteralstrong{\sphinxupquote{bboxes\_3d}} (\sphinxstyleliteralemphasis{\sphinxupquote{dict}}) \textendash{} list of bounding box information with pseudo\sphinxhyphen{}3d image coordinate frame

\item {} 
\sphinxstyleliteralstrong{\sphinxupquote{pc\_size}} (\sphinxstyleliteralemphasis{\sphinxupquote{float}}) \textendash{} percentage of the size of the bounding box {[}0.0 1.0{]}

\end{itemize}

\item[{Returns}] \leavevmode
camera image with the correspondent bounding boxes

\item[{Return type}] \leavevmode
np.array

\end{description}\end{quote}

\end{fulllineitems}


\end{fulllineitems}



\chapter{Indices and tables}
\label{\detokenize{index:indices-and-tables}}\begin{itemize}
\item {} 
\DUrole{xref,std,std-ref}{genindex}

\item {} 
\DUrole{xref,std,std-ref}{modindex}

\item {} 
\DUrole{xref,std,std-ref}{search}

\end{itemize}


\renewcommand{\indexname}{Python Module Index}
\begin{sphinxtheindex}
\let\bigletter\sphinxstyleindexlettergroup
\bigletter{r}
\item\relax\sphinxstyleindexentry{radiate}\sphinxstyleindexpageref{radiate:\detokenize{module-radiate}}
\end{sphinxtheindex}

\renewcommand{\indexname}{Index}
\printindex
\end{document}